% !TEX TS-program = pdflatex
% !TEX encoding = UTF-8 Unicode

% This is a simple template for a LaTeX document using the "article" class.
% See "book", "report", "letter" for other types of document.

% align equations to the left
% use larger type; default would be 10pt
\documentclass[fleqn, 11pt]{article}

\usepackage[utf8]{inputenc} % set input encoding (not needed with XeLaTeX)

%%% Examples of Article customizations
% These packages are optional, depending whether you want the features they
% provide.
% See the LaTeX Companion or other references for full information.

%%% PAGE DIMENSIONS
\usepackage{geometry} % to change the page dimensions
\geometry{a4paper,top=20mm,bottom=20mm}
% for example, change the margins to 2 inches all round
% \geometry{margin=2in}
% set up the page for landscape
% \geometry{landscape}
%   read geometry.pdf for detailed page layout information

% support the \includegraphics command and options
\usepackage{graphicx}
\usepackage{wrapfig}

% Activate to begin paragraphs with an empty line rather than an indent
\usepackage[parfill]{parskip}

%%% PACKAGES
% for much better looking tables
\usepackage{booktabs}
% for better arrays (eg matrices) in maths
\usepackage{array}
% very flexible & customisable lists (eg. enumerate/itemize, etc.)
\usepackage{paralist}
% adds environment for commenting out blocks of text & for better verbatim
\usepackage{verbatim}
% make it possible to include more than one captioned figure/table in a single
% float
\usepackage{subfig}
% These packages are all incorporated in the memoir class to one degree or
% another...

%%% HEADERS & FOOTERS
\usepackage{fancyhdr} % This should be set AFTER setting up the page geometry
\pagestyle{fancy} % options: empty , plain , fancy
\renewcommand{\headrulewidth}{0pt} % customise the layout...
\lhead{}\chead{}\rhead{}
\lfoot{}\cfoot{\thepage}\rfoot{}

%%% SECTION TITLE APPEARANCE
\usepackage{sectsty}
% (See the fntguide.pdf for font help)
\allsectionsfont{\sffamily\mdseries\upshape}
% (This matches ConTeXt defaults)

%%% ToC (table of contents) APPEARANCE
% Put the bibliography in the ToC
\usepackage[nottoc, notlof, notlot]{tocbibind}
% Alter the style of the Table of Contents
\usepackage[titles, subfigure]{tocloft}
\renewcommand{\cftsecfont}{\rmfamily\mdseries\upshape}
\renewcommand{\cftsecpagefont}{\rmfamily\mdseries\upshape} % No bold!

\usepackage{calc}
\usepackage{lmodern}
\usepackage{amssymb}
\usepackage{amsmath}
\usepackage{mathdots}
\usepackage{mathtools}
\usepackage{enumerate}

\usepackage{algorithm,algorithmicx}
\usepackage[noend]{algpseudocode}
\algrenewcommand\alglinenumber[1]{{\sf\footnotesize\texttt{#1}}}

\usepackage{xcolor}
\renewcommand\thefootnote{\textcolor{blue}{\arabic{footnote}}}

\renewcommand{\figurename}{Fig.}
\def\is{\coloneqq}

\usepackage[linguistics]{forest}

\overfullrule=2em

%%% END Article customizations

%%% The "real" document content comes below...

\title{Algoritmos y Estructuras de Datos 3: Gloppi Ya}
\author{Ignacio E. Losiggio (751/17), Federico A. Sabatini (579/17)}
% Activate to display a given date or no date (if empty), otherwise the current
% date is printed
%\date{}

\begin{document}

\maketitle

\tableofcontents

\section{Introducción}

El problema de la mochila consiste en elegir de un conjunto de ítems con peso y
valor los necesarios para llenar una mochila de acuerdo al peso máximo que
puede tolerar buscando al mismo tiempo maximizar el valor de los objetos
elegidos.

\subsection{Formulación del problema}

Dado un espacio de tamaño $W$ (la mochila), y un conjunto de ítems $(x_1, x_2,
\dots, x_n)$ con pesos $(p_1, p_2 \dots x_n)$ y valores $(v_1, v_2 \dots v_n)$
ambos enteros no negativos, hallar un subconjunto de índices que maximice el
valor total sin exceder el peso máximo $W$, es decir:

\begin{enumerate}
	\item $\sum_{\{i \in \text{índices}\}} v_i$ es la respueta
	\item $\sum_{\{i \in \text{índices}\}} p_i \le W$
	\item La respuesta es máxima (no existe otro conjunto que tenga un
	valor mayor y  pese menor o igual a $W$).
\end{enumerate}

\subsection{Ejemplos}

$W = 30$, ítems $=$
\begin{tabular}{l l}
	Peso  & Valor \\ \toprule
	20 & 50 \\
	10 & 5  \\
	30 & 35
\end{tabular} \\
\textbf{Respuesta:} 55 \\
\textbf{Subconjuntos solución:} $\{(20,50), (10,5)\}$

$W = 60$, ítems $=$
\begin{tabular}{l l}
	Peso & Valor \\ \toprule
	45 & 150 \\
	20 & 50 \\
	20 & 40 \\
	20 & 60
\end{tabular} \\
\textbf{Respuesta:} 150 \\
\textbf{Subconjuntos solución:} $\{(45, 150)\}$ ó $\{(20, 50), (20, 40), (20,
60)\}$

Como puede notarse, es posible que exista más de un subconjunto máximo.

\subsection{Pautas de diseño}

El objetivo de este trabajo es mostrar diferentes algoritmos que resuelven el
problema planteado comparando sus ventajas y desventajas. Para esto se
construirán diferentes experimentos buscando respaldar los resultados teóricos
con experimentos apropiados. Por cada algoritmo ofreceremos cotas de
complejidad tanto en tiempo como en memoria.

\section{Algoritmos explorados}

\subsection{Fuerza bruta}

Dado un conjunto de ítems, el máximo valor que cumple la condición de no
exceder el peso máximo de $W$, existe entre uno (o varios) de los subconjuntos
posibles.

Si no conocemos ninguna forma buena de eliminar subconjuntos inútiles entonces
el número total de subconjuntos a revisar es el tamaño del conjunto de partes
de nuestros ítems ($2^n$).

Podemos ofrecer como primer acercamiento un algoritmos iterativo que recorra
todos los subconjuntos posibles, calculando peso y valor y quedándose con el
mejor.

\begin{algorithm}
\caption{Knapsack con fuerza bruta}
\begin{algorithmic}[1]
\Statex
\Function{Knapsack-Dinamica}{$W, \text{items}$}
\State $\text{mejor} \is 0$
\State $c \is 1$
\While{$c < 2^{\#items}$}
	\State $s \is \Call{calcular}{c, \text{items}}$
	\If{$s_{peso} \leq W \land s_{valor} > \text{mejor}$}
		\State mejor $\is s_{valor}$
	\EndIf
	\State $c \is c + 1$
\EndWhile
\State \Return mejor
\EndFunction
\end{algorithmic}
\begin{description}
	\item[\textbf{Complejidad algorítmica:}] $O(n \times 2^n)$
	\item[\textbf{Memoria requerida:}] $O(n)$
\end{description}
\end{algorithm}

En cada ciclo generamos las sumas $s_{valor}$ y $s_{peso}$ del $c$ésimo
conjunto\footnote{$c$ es el índice del conjunto en la enumeración utilizada por
\textsc{calcular}}. Entonces, una enumeración posible es representar la
pertenencia del $i$ésimo ítem con el $i$ésimo dígito de la repres entación en
base 2 de $c$\footnote{Es decir: $items[i] \in \text{conjunto} \iff c \mod 2^i
= 1$.}.

La forma más simple de calcular las sumas es $O(n)$\footnote{Ya que hay que
revisar por cada uno de los $n$ ítems si se encuentra en el conjunto para
sumarlos.}.  Dado que el $c$ recorre todos los valores en el intervalo $[1,2^n
- 1]$ esta suma se calcula $O(2^n)$ veces ofreciéndonos la complejidad mostrada
de $O(n \times 2^n)$.

\subsection{Backtracking}

Otra estrategia consiste en el uso de un algoritmo exploratorio dónde por cada
elemento del problema consideraremos agregarlo o no y buscaremos la mejor
mochila en cada caso.

Repetido el proceso por $n$ elementos, la cantidad de ramas del problema crece
de forma exponencial, con un total de $2^n$ mochilas. Siempre y cunado la
complejidad de evaluar cada mochila esté en $O(1)$ la complejidad final no
excederá $O(2^n)$.

La ventaja de un algoritmo de backtracking es que si podemos asegurar con
antelación que una familia de soluciones no es viable podemos evitar
explorarla. Dichas estrategias se denominarán podas del problema a partir de
ahora. Veremos 2 algoritmos que implementan podas en sus búsquedas.

\subsubsection{Argumentos utilizados por los algoritmos}

Para simplificar la escritura de los algoritmos de backtracking utilizamos un
argumento $s$ que codifica la ``solución parcial'' a mejorar en cada etapa del
algoritmo. 

\subsubsection{Poda por factibilidad}

Nuestra primera poda requiere un orden previo de los items (por peso de forma
creciente) esto nos supone un coste inicial de $O(n \times \log n)$. Gracias al
orden podemos asegurar que si para un índice $i$ el $i$-ésimo ítem no puede
agregarse a $s$ entonces todos los ítems con índice mayor tampoco pueden (por
estar ordenados).

Es sencillo notar en esta poda que no descartamos soluciones óptimas dado que
solo se descartan al exceder el peso (y no hay pesos negativos).

El otro criterio de corte que utilizamos es más simple: Si todo entra en la
mochila, todo va en la mochila. Dado que no hay valores negativos agregar a la
mochila siempre mejora la solución.

\clearpage

\begin{algorithm}
\caption{Backtracking con poda por factibilidad}
\begin{algorithmic}[1]
\item[\textbf{Inicialización:}]
\item[] \begin{itemize}
	\item[] $p$ los ítems ordenados de menor a mayor según su peso
	\item[] $W$ el peso máximo de la mochila
	\item[] $s$ (solución parcial)
	\begin{itemize}
		\item[] $s_i = 0$ índice actual
		\item[] $s_{valor} = s_{peso} = 0$ (la mochila vacia \O)
		\item[] $s_{pesosCalculados} = \sum_{i = 0}^{\#p - 1} p[i]_{peso}$
	\end{itemize}
\end{itemize}
\Statex
\Function{Knapsack-Factibilidad}{$p, W, s$}
\If{$s_i = \#p$}
	\Comment Caso base
	\State \Return $s_{valor}$
\EndIf
\If{$s_{peso} + s_{pesosCalculados} \leq W$}
	\Comment Agrego todo a la mochila
	\State \Return $s_{valor} + \sum_{i=s_i}^{\#p - 1} p[i]_{valor}$
\EndIf
\If{$s_{peso} + p[s_i]_{peso} > W $}
	\Comment No puedo sumar ningún valor más
	\State \Return $s_{valor}$
\EndIf

\State $\text{con-valor} \is \Call{Knapsack-Factibilidad}{p, W, \text{ considerar } s_i}$
\Comment $s_i \in$ mochila
\State $\text{sin-valor} \is \Call{Knapsack-Factibilidad}{p, W, \text{ no considerar } s_i}$
\Comment $s_i \not\in$ mochila
\State \Return $\Call{max}{\text{con-valor}, \text{sin-valor}}$
\EndFunction
\end{algorithmic}
\begin{description}
	\item[\textbf{Complejidad algorítmica:}] $O(2^n)$
	\item[\textbf{Memoria requerida:}] $O(n)$
\end{description}
\end{algorithm}

En este algoritmo evaluar si una familia de mochilas se excede de peso depende
de una simple comparación (\texttt{6}). Considerar y no considerar los
elementos (\texttt{8})(\texttt{9}) se implementa construyendo la nueva solución
parcial\footnote{Se aumenta el índice, se actualizan el valor, peso y los pesos
restantes según sea necesario.} a través de una cantidad acotada de sumas y
restas.

El otro criterio de corte (\texttt{5}) no modifica la complejidad dado que
cuando se cumple la condición se pueden considerar todos los elementos
siguientes. Ésto es más barato que calcular ambas recursiones.

Los casos favorables de esta poda se encuentran cuando $W$ resulta mucho más
pequeño o grande que los pesos de todos los ítems, dado que se puede evitar el
analisis de mochilas inviables (\texttt{6}) o rellenar a la mochila sin
considerar los pesos individuales (\texttt{4}).

\subsubsection{Poda por optimalidad}

La segunda poda que veremos no requiere un ordenamiento previo, en su lugar
debe calcularse $\sum_{i = 0}^{\#p - 1}p[i]_{valor}$ la suma de los valores y
ofrecer un candidato ($m$) el cual debe ser una mochila válida.

Dado que podemos con antelación considerar todos los elementos restantes; si el
valor de agregarlos no excede al candidato $m$ entonces esa familia de mochilas
no será óptima y la podamos.

A la hora de considerar cada ítem podemos utilizar la mochila candidato $m$ y
el valor de nuestros hijos para determinar si somos capaces de
mejorarla\footnote{En caso contrario podemos devolver $m$ ya que sabemos que es
una mochila válda.}. En nuestra implementación previo a correr el algoritmo
ordenamos los ítems para conseguir un $m$ grande lo antes posible.

\clearpage

\begin{algorithm}
\caption{Backtracking con poda por optimalidad}
\begin{algorithmic}[1]
\item[\textbf{Inicialización:}]
\item[] \begin{itemize}
	\item[] $p$ los ítems (preferiblemente ordenados por valor)
	\item[] $W$ el peso máximo de la mochila
	\item[] $s$ (solución parcial)
	\begin{itemize}
		\item[] $s_i = \#p - 1$
		\item[] $s_{valor} = s_{peso} = 0$
		\item[] $s_{valoresRestantes} = \sum_{i = 0}^{\#p - 1} p[i]_{valor}$
	\end{itemize}
	\item[] $m = 0$ la mejor mochila encontrada hasta el momento
\end{itemize}
\Statex
\Function{Knapsack-Optimalidad}{$p, W, m, s$}
\If{$s_i = -1$}
	\Comment Caso base
	\State \Return m
\EndIf
\If{$s_{valor} + s_{valoresRestantes} \leq \text{mejor}$}
	\Comment No puedo mejorar
	\State \Return $m$
\EndIf
\If{$s_{peso} + p[s_i]_{peso} \leq W$}
	\State $m \is \Call{Knapsack-Optimalidad}{p, W, m, \text{ considerar } s_i}$
	\Comment $s_i \in$ mochila
\EndIf
\State $m \is \Call{Knapsack-Optimalidad}{p, W, m, \text{ no considerar } s_i}$
\Comment $s_i \not\in$ mochila
\State \Return $m$
\EndFunction
\end{algorithmic}
\begin{description}
	\item[\textbf{Complejidad algorítmica:}] $O(2^n)$
	\item[\textbf{Memoria requerida:}] $O(n)$
\end{description}
\end{algorithm}

Las podas son similares a las anteriores pero en este caso sólo requieren
operaciones elementales (sumas y comparaciones). La operación considerar
difiere en qué propiedades de la mochila actualiza.

\subsection{Meet in Middle}

También podemos encarar el problema con una técnica de búsqueda diferente:
dividimos el problema en 2 partes iguales de dimensión $\frac{n}{2}$,
conseguimos suficiente información para poder resolver nuestro problema y
finalmente construimos la respuesta.

Si llamamos $A$ y $B$ a las dos mitades de nuestro conjunto de ítems y
construimos sus conjuntos de partes entonces los conjuntos solución son todos
de la forma $a \bigcup b$\footnote{Si una solución se encontrase completa en
cualquiera de las dos mitades se puede unir al conjunto vacío de la otra.}. Una
vez construidos podemos buscar la mejor pareja de cualquier subconjunto de una
mitad dentro de los subconjuntos de la otra.

Definiremos entonces que nuestro algoritmo debe realizar los siguientes pasos:

\begin{enumerate}
\item Separar el conjunto de ítems en 2 subconjuntos $A$, $B$. $A$ teniendo los
primeros $\frac{n}{2}$ ítems y $B$ el resto.

\item Construir las mochilas posibles con $A$ y con $B$. Es decir los pares
$(peso, valor)$ de cada subconjunto. El precio de armarlas es
$O(2^{\frac{n}{2}})$.

\item Combinarlas: encontrar las combinaciones $a \bigcup b$ quedándonos con la
mejor de todas. Existen multiples formas de hacerlo:

\begin{enumerate}
	\item Iterar sobre todos los elementos (desestimaremos esta posibilidad
	dado que su complejidad sería $O(2^n)$, no ganaríamos mucho respecto de
	fuerza bruta).

	\item Ordenar $A$ ó $B$. Descartar conjuntos que no sean óptimos en el
	ordenado. Recorrer el otro conjunto linealmente buscando para cada
	elemento su mejor pareja (utilizando una búsqueda binaria en el
	ordenado). Este método nos ofrece una complejidad final de $O(n \times
	2^{\frac{n}{2}})$.
\end{enumerate}
\end{enumerate}

\begin{algorithm}
\caption{Meet in the middle}
\begin{algorithmic}[1]
\item[\textbf{Inicialización:}]
\item[] \begin{itemize}
	\item[] $p$ conjunto de pares inicial.
	\item[] $p[i]_{peso}$ i-ésimo peso del par p
	\item[] $p[i]_{valor}$ i-ésimo valor del par p
	\item[] $v_{mejor} = 0$
\end{itemize}
\item[\textbf{Funciones auxiliares:}]
\item[] \begin{itemize}
	\setlength\itemsep{0.3em}
	\item[] \Call{mochilas}{$p$} Arma el conjunto de mochilas posibles con
	los ítems de $p$.
	
	\item[] \Call{ordenar}{$p$} Ordenamiendo por comparación con
	complejidad $O(\#p\log \#p)$
	
	\item[] \Call{descartar}{$p$} Elimina el $p_i$ si $p_{i-1}$ posee un
	valor mayor con menor peso. Complejidad $O(\#p)$.

	\item[] \Call{buscar}{$p, k$} Búsqueda binaria, devuelve la última
	mochila de peso $k$ o la última con un peso menor.
\end{itemize}
\Statex
\Function{Knapsack-MeetInTheMiddle}{$p, w$}
\State $Izq$ $\is \Call{mochilas}{p[0:\frac{n}{2}]}$
\State $Der$ $\is \Call{mochilas}{p[\frac{n}{2}:n]}$
\Comment Armo los subconjuntos

\State $Der$ $\is \Call{ordenar}{\text{der}}$
\Comment Ordeno uno de los 2 subconjuntos
\State $Der$ $\is \Call{descartar}{\text{der}}$
\Comment Elimino no óptimos

\For{$(a \in Izq)$}
	\If{$a_{peso} \leq w$}
	\State $b \is \Call{buscar}{Der, w - a_{peso}}$
		\If{$a_{peso} + b_{peso} > v_{mejor}$}
			\State  $v_{mejor} \is a_{peso} + b_{peso} > $
		\EndIf
	\EndIf
\EndFor
\State \Return $v_{mejor}$
\EndFunction
\end{algorithmic}
\begin{description}
	\item[\textbf{Complejidad algorítmica:}] $O(n \times 2^{\frac{n}{2}})$
	\item[\textbf{Memoria requerida:}] $O(2^{\frac{n}{2}})$
\end{description}
\end{algorithm}

Analicemos la complejidad, \Call{Ordenar}{der} es $O(2^\frac{n}{2} \times
\log2^\frac{n}{2})$ ya que el conjunto a ordenar es partes de $A$ ó $B$.
$O(2^\frac{n}{2} \times \log2^\frac{n}{2})$ es equivalente a $O(n \times
2^{\frac{n}{2}})$.

\Call{Descartar}{der} tras ordenar es lineal sobre el tamaño del conjunto
$O(2^\frac{n}{2})$. Sólo debemos comparar el conjunto actual contra el anterior
para descartarlo\footnote{Un conjunto de peso $x$, se considera no óptimo si
tiene peso mayor a $x$ y no lo supera en valor final.}, eliminación que es
vital para aplicar búsqueda binaria\footnote{No es necesario eliminar el
elemento, basta copiar el valor del elemento anterior.}.

Finalmente recorrer el conjunto no ordenado linealmente es $O(2^\frac{n}{2})$ y
por cada conjunto se debe encontrar su mejor pareja, dado que el otro conjunto
es ordenado ésta complejidad es $O(\log{2^\frac{n}{2}})$ que es equivalente a
$O(n)$ dándonos nuevamente la complejidad $O(n \times 2^{\frac{n}{2}})$.

\subsection{Programación dinámica}

El algoritmo de programación dinámica hace uso de las técnicas exploratorias de
backtracking. A partir de un elemento realizaremos dos llamados recursivos
considerando si agregarlo o no. La diferencia que lo distingue de backtracking
es que almacenaremos las decisiones tomadas para no tener que volver a
calcularlas.

Nuestros subproblemas son las mochilas con los primeros $i$ elementos ($i \le
\#\text{items}$) y un peso máximo $w'$ ($w' \le W$). Podemos notar que si
tenemos todas las mochilas $(i' < i, w'' \le w')$ entonces nuestro problema se
reduce a determinar si el mayor valor se obtiene utilzando el $i$-ésimo
elemento (dado que no se exceda en peso) o ignorándolo.

Construiremos un algoritmo \emph{top-down} (consideramos el problema completo y
lo dividimos hasta llegar a problemas triviales) con dos partes, un algoritmo
recursivo simple y la memoización de sus resultados intermedios. Para esto
optaremos como estructura inicializar una matriz, de $n$ filas con $W$
columnas.

Finalmente una vez reducido el problema a la construcción de una mochila de $0$
elementos está claro que el mejor valor posible es $0$. Este será nuestro caso
base.

\begin{algorithm}
\caption{Knapsack con programación dinámica}
\begin{algorithmic}[1]
\item[\textbf{Inicialización:}]
\item[] \begin{itemize}
	\item[] $i = \#p - 1$
	\item[] $\text{restante} = W$
	\item[] $p$ la lista de ítems a considerar
	\item[] $mem[n][W]$ inicialmente $\bot$ en cada celda
\end{itemize}
\Statex
\Function{Knapsack-Dinamica}{$i, \text{restante}, p, mem$}
\If{$i < 0$}
	\State \Return $0$
\Comment Casos base
\EndIf
\If{$mem[i][\text{restante}] \neq \bot$}
	\State	\Return $mem[i][\text{restante}]$
	\Comment Memoización
\EndIf
\State $\text{sin} \is \Call{Knapsack-Dinamica}{i - 1, r, p}$
\Comment $p[i] \not \in \text{mochila}$
\If{$p[i]_{peso} \le \text{restante}$}
	\State $\text{con} \is p[i]_{valor} + \Call{Knapsack-Dinamica}{i - 1, r - p[i]_{peso}, p}$
	\Comment $p[i] \in \text{mochila}$
\Else
	\State $\text{con} \is 0$
\EndIf
\State $mem[i][\text{restante}] \is \Call{max}{\text{sin}, \text{con}}$
\Comment Guardo el mejor de ambos
\State \Return $mem[i][\text{restante}]$
\EndFunction
\end{algorithmic}
\begin{description}
	\item[\textbf{Complejidad algorítmica:}] $O(n \times W)$
	\item[\textbf{Memoria requerida:}] $O(n \times W)$
\end{description}
\end{algorithm}

En un principio inicializamos $mem$ como $\bot$ lo que cuesta $O(n \times W)$.
Por cada valor de $i$ no trivial se realizarán a lo sumo $2i$ llamadas
recursivas, si $i > W$ tendremos superposición de subproblemas\footnote{Dado
que no realizamos llamadas con peso negativo y que lo único que diferencia un
subproblema de otro es el par $(i, w')$.} por lo que por cada $i$ realizaremos
a lo sumo $W$ cálculos. Entonces para cada $i$ tenemos una complejidad de
$O(W)$ y $\sum_{i=0}^{n} O(W) = O(n \times W)$.

\clearpage

La correctitud de este algoritmo depende de dos variables: tener un algoritmo
de ``backtracking'' correcto y no perder información al memoizar. Lo segundo
puede notarse dado que la clave con la que memoizamos depende de los únicos
parámetros que cambian en cada llamada recursiva\footnote{Nuestro algoritmo
sólo depende de sus parámetros de entrada, por lo que es correcto memoizarlo.}.
Lo primero es más complejo, pero comparando la estructura con la de
$\Call{Knapsack-Factibilidad}{p, W, s}$ se puede ver que el algoritmo de
backtracking subyacente es una variación de éste\footnote{Dónde se recorren los
ítems al revés y no se corta si entran todos los ítems próximos en la mochila.}

\begin{figure}[h]
\begin{center}
\includegraphics[width=\textwidth]{{fotos/pdf.dinamica}.pdf}
\caption{Llamadas recursivas sobre \texttt{casos/pdf.in}. Las líneas punteadas
indican valores ya calculados.}
\end{center}
\end{figure}

\section{Experimentación}

\begin{itemize}
\item Se repite cada prueba 50 veces. Se calcula el promedio.

\item Se dispone de un tiempo límite de 30 segundos por cada prueba (es decir,
el máximo tiempo de procesamiento por un caso de prueba es 25 minutos).

\item Cada experimentación consta de 250 casos de prueba.

\item Los parámetros libres en cada prueba poseen distribución
uniforme\footnote{Es decir, si en el resumen de parámetros dice $n \in [15;
24]$ entonces $n$ se genera con $\lfloor U(15, 25)\rfloor$. Se utiliza la misma
técnica para todos los parámetros libres.}.
\end{itemize}

\clearpage

\subsubsection{Experimento A}

\begin{wrapfigure}{R}[2cm]{0.45\linewidth}
\vspace{-50pt}
\includegraphics[width=\linewidth]{{fotos/exp.a.fuerza_bruta}.pdf}
\caption{Fuerza bruta}
\includegraphics[width=\linewidth]{{fotos/exp.a.backtracking_fact}.pdf}
\caption{Backtracking (factibilidad)}
\includegraphics[width=\linewidth]{{fotos/exp.a.backtracking_opt}.pdf}
\caption{Backtracking (optimalidad)}
\includegraphics[width=0.45\textwidth]{{fotos/exp.a.dinamica}.pdf}
\caption{Programación dinámica}
\end{wrapfigure}

Nuestro primer experimento tiene como objetivo ver cómo se relaciona la
cantidad de ítems a procesar con el tiempo que tarda cada algoritmo en
procesarlos. Para este fin usaremos una entre 15 y 24 ítems y para evitar los
casos favorables de las podas de backtracking el peso máximo será la mitad del
peso total a procesar. Cada peso y valor estará entre 0 y 100.

\begin{center}
$n \in [15; 24]; W=\frac{\sum p_i}{2}; 0 \leq p_i, v_i \leq 100$
\end{center}

Los resultados fueron los esperables para casi todos los algoritmos exceptuando
el caso de la poda de optimalidad. Es probable que ésta esté podando de manera
muy agresiva haciendo que la complejidad promedio esté alrededor de
$2^\frac{n}{2}$. El gráfico del algoritmo de programación dinámica es
particular pero dado que su complejidad depende tanto de $n$ como de $W$ no se
puede plotear mucho mejor\footnote{Probablemente una serie de boxplots hubiera
sido lo más acertado para ese gráfico en particular.}.

\begin{center}
\includegraphics[width=0.45\textwidth]{{fotos/exp.a.mitm}.pdf}
\captionof{figure}{Meet in the Middle}
\end{center}

\subsubsection{Experimento B}

Para analizar más a fondo los casos favorables y desfavorable de nuestros
algoritmos de backtracking construimos tests variando $n$ y $W$. $W$ es un
valor aleatorio menor o igual a la suma de todos los pesos (un $W$ muy alto es
trivial de resolver para las podas contruídas).

\begin{center}
$n \in [25; 70]; W \in [2; \sum p_i]; 0 \leq p_1, v_i \leq 100$
\end{center}

\clearpage

Cada punto en los gráficos a continuación es el promedio de 50 corridas de un
caso de prueba. Los puntos grises son casos que no pudimos evaluar debido al
tiempo límite de 30 segundos por corrida. El script que genera estos gráficos
puede encontrarse  en \texttt{scripts/experimento\_b.plot}.

\vspace{1em}
\hspace{-3cm}
\begin{minipage}{1.4\textwidth}
\begin{minipage}{0.33\textwidth}
\centering
\includegraphics[width=\linewidth]{{fotos/exp.b.backtracking_fact}.pdf}
\captionof{figure}{Backtracking (factibilidad)}
\end{minipage}
\begin{minipage}{0.33\textwidth}
\centering
\includegraphics[width=\linewidth]{{fotos/exp.b.backtracking_opt}.pdf}
\captionof{figure}{Backtracking (optimalidad)}
\end{minipage}
\begin{minipage}{0.33\textwidth}
\centering
\includegraphics[width=\linewidth]{{fotos/exp.b.mitm}.pdf}
\captionof{figure}{Meet in the Middle}
\end{minipage}
\end{minipage}

Puede verse que tanto la poda por factibilidad cómo la poda por optimalidad
logran resolver valores de $n$ dónde meet-in-the-middle se queda sin memoria o
tiene que operar con subconjuntos demasiado grandes. Sin embargo sólo son
capaces de resolver éstos problemas para combinaciones de $n$ y $W$ dónde la
solución no requiera explorar mochilas de demasiados elementos.

\subsubsection{Experimento C}

Como el experimento A resultó muy pequeño para mostrar el comportamiento del
algoritmo de programación dinámica decidimos construir casos de test mucho más
grandes. Dado que la complejidad teórica es $O(n \times W)$ se consideró que al
experimentar con $n = W$ sería razonable esperar un crecimiento cuadrático.

\begin{center}
$n = W \in [100; 1000]; 0 < p_i < \frac{W}{2}; 1 \leq v_i \leq 5000$
\end{center}

\begin{figure}[h]
\includegraphics{{fotos/exp.c.dinamica}.pdf}
\caption{Programación Dinámica}
\end{figure}

\clearpage

\subsubsection{Comparación de los algoritmos}
Por último se decidió usar el set del experimento A para hacer una comparación entre las velocidades efectivas que a los algoritmos les tomó realizar el experimento.

\begin{figure}[h]
	\includegraphics{{fotos/exp.a.algos_todos}.pdf}
	\caption{Velocidad de los algoritmos}
\end{figure}

los resultados no fueron los esperados en backtracking con poda por optimalidad ya que mostro una velocidad muy superior a la teórica, consideramos que este comportamiento se debió a que los casos del experimento A fueron muy favorables permitiendole construir un buen candidato a mejor mochila rapidamente y podando el resto. Consideraremos este resultado fuera de lo normal, y lo desestimaremos.

\section{Conclusión}

Hemos considerado cómo resolver el problema de la mochila con diferentes
métodologias, así como hemos buscado identificar la complejidad en tiempo y
memoria requerida y aplicado heúristicas que busquen podar y reducir el espacio
de nuestro problema.

Tras someter a nuestros algoritmos a extensas pruebas. Los resultados mas
destacables son los siguientes:

\begin{enumerate}
	\item La poda por optimalidad puede construír una buena mochila con la
	que podar rápidamente, gracias a esto muestra un rendimiento superior a
	Meet-in-the-Middle en las condiciones
	adecuadas\footnote{Meet-in-the-Middle fracasan rotundamente dado un $n$
	suficientemente largo pero la poda por optimalidad puede construir una
	solución en ciertas situaciones.} pese a que éste tenga una complejidad
	teórica mejor.

	\item El algoritmo de memoria dinámica tuvo un rendimiento
	extremadamente rápido en comparación con los otros, mantuvo un consumo
	de memoria menor que Meet-in-the-Middle aunque mayor que los algoritmos
	de backtracking que realizan\\búsquedas exploratorias y poseen un
	consumo lineal.
	
	\item La poda por factibilidad, mostró ser poco efectiva en casos de
	$n$ grande, lo cual se debe en gran medida a que ésta se realiza en las
	ramas más bajas del árbol de ejecución. La poda solo muestra
	efectividad para casos en los que $W$ es extremadamente bajo o
	excesivamente alto respecto del peso de los ítems.
\end{enumerate}

\subsection{Trabajo futuro}

A continuación se enumerarán una lista de tareas pendientes.
\begin{enumerate}
\item Los resultados de nuestros experimentos no reflejaron la velocidad teórica de backtracking por optimalidad y deberíamos crear nuevos experimentos que sometan a este algoritmo a un escenario desfavorable para acercarnos más a su cota de complejidad teórica.

\item Programación dinámica fue sometído a pruebas mucho más exigentes que el resto de los algoritmos, sin embargo no se realizarón pruebas sobre sus casos desfavorables, se necesitaría analizar que sucede en casos donde $W$ es excesivamente más grande que $N$. Por ejemplo si $n^2 = W$ esto implicaría que tanto los costos de memoría requerida como complejidad se volvieron $n^3$.



\end{enumerate}



\section{Referencias}
\begin{itemize}
	\item[] \textit{Computing partitions with applications to the knapsack
	- Ellis Horowitz and Sartaj Sahni.}
\end{itemize}

\end{document}
