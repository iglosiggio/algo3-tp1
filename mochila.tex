% !TEX TS-program = pdflatex
% !TEX encoding = UTF-8 Unicode

% This is a simple template for a LaTeX document using the "article" class.
% See "book", "report", "letter" for other types of document.

% align equations to the left
% use larger type; default would be 10pt
\documentclass[fleqn, 11pt]{article}

\usepackage[utf8]{inputenc} % set input encoding (not needed with XeLaTeX)

%%% Examples of Article customizations
% These packages are optional, depending whether you want the features they
% provide.
% See the LaTeX Companion or other references for full information.

%%% PAGE DIMENSIONS
\usepackage{geometry} % to change the page dimensions
\geometry{a4paper} % or letterpaper (US) or a5paper or....
% for example, change the margins to 2 inches all round
% \geometry{margin=2in}
% set up the page for landscape
% \geometry{landscape}
%   read geometry.pdf for detailed page layout information

% support the \includegraphics command and options
\usepackage{graphicx}

% Activate to begin paragraphs with an empty line rather than an indent
\usepackage[parfill]{parskip}

%%% PACKAGES
% for much better looking tables
\usepackage{booktabs}
% for better arrays (eg matrices) in maths
\usepackage{array}
% very flexible & customisable lists (eg. enumerate/itemize, etc.)
\usepackage{paralist}
% adds environment for commenting out blocks of text & for better verbatim
\usepackage{verbatim}
% make it possible to include more than one captioned figure/table in a single
% float
\usepackage{subfig}
% These packages are all incorporated in the memoir class to one degree or
% another...

%%% HEADERS & FOOTERS
\usepackage{fancyhdr} % This should be set AFTER setting up the page geometry
\pagestyle{fancy} % options: empty , plain , fancy
\renewcommand{\headrulewidth}{0pt} % customise the layout...
\lhead{}\chead{}\rhead{}
\lfoot{}\cfoot{\thepage}\rfoot{}

%%% SECTION TITLE APPEARANCE
\usepackage{sectsty}
% (See the fntguide.pdf for font help)
\allsectionsfont{\sffamily\mdseries\upshape}
% (This matches ConTeXt defaults)

%%% ToC (table of contents) APPEARANCE
% Put the bibliography in the ToC
\usepackage[nottoc, notlof, notlot]{tocbibind}
% Alter the style of the Table of Contents
\usepackage[titles, subfigure]{tocloft}
\renewcommand{\cftsecfont}{\rmfamily\mdseries\upshape}
\renewcommand{\cftsecpagefont}{\rmfamily\mdseries\upshape} % No bold!

\usepackage{calc}
\usepackage{lmodern}
\usepackage{amssymb}
\usepackage{amsmath}
\usepackage{mathdots}

\overfullrule=2em

%%% END Article customizations

%%% The "real" document content comes below...

\title{Algoritmos y Estructuras de Datos 3: Gloppi Ya}
\author{Ignacio E. Losiggio, Federico A. Sabatini}
% Activate to display a given date or no date (if empty), otherwise the current
% date is printed 
%\date{}

\begin{document}

\maketitle
\section{Presentación del problema Knapsack} 

\subsubsection{Dado un espacio de tamaño W, y un conjunto de tuplas $(x_i,y_i)$ con pesos $(x_1,x_2..x_n)$ y precios $(y_1,y_2..y_n)$ ambos enteros no negativos,
hallar un subconjunto de indices que maximiza un valor $z_{out} \in\mathbb{E} \geq 0$, de la siguiente forma:}

\begin{itemize}
	\item $z_{out} $ = max $\sum \{ y_i : i \in indices \} $ (subconjunto cuya sumatoria maximiza el precio)
	\item $\sum \{ x_i : i \in indices \} \le W $ (subconjunto cuyos pesos no exceda el tamaño de W)
\end{itemize}

\subsubsection{Se plantean los siguientes ejemplos del problema:}

A) - 
\begin{tabular}{l l}
	tamaño (size) & precio \\ \toprule
    20 & 50 \\
    10 & 5  \\
    10 & 30 \\
    8 & 35  \\
\end{tabular}
y un espacio W de tamaño 20. \\

Res: $z_{out}$ es 65 y el subconjunto que maximiza el valor son las tuplas [(10,30) y (8,35)] \\

B) - 
\begin{tabular}{l l}
	tamaño (size) & precio \\ \toprule
    25 & 10 \\
    20 & 5 \\
    2  & 1 \\
    1  & 1 \\
    1  & 1 \\
    1  & 1 \\
\end{tabular}
y un espacio W de tamaño 25. \\

Res: $z_{out}$ es 10 y el subconjunto que maximiza el valor es la tupla [(25,10)] \\

C) - 
\begin{tabular}{l l}
	tamaño (size) & precio \\ \toprule
    30 & 100 \\
    25 & 25  \\
    22 & 15  \\
    22 & 40  \\
    26 & 50  \\
    24 & 50  \\

\end{tabular}
y un espacio W de tamaño 50. \\

Res: $z_{out}$ es 100 y existen 2 subconjuntos que maximizan el valor, el subconjunto [(30,100)] y [(26,50), (24,50)]

\subsubsection{Presentados diferentes casos del problema se plantean diferentes preguntas: } 

.\\
¿Cómo sería la resolución de un algoritmo que cumpla con dicho problema?  \\
¿Existen algoritmos que tarden menos que otros?  \\
¿Cómo afectan la variable W y la cantidad de N-tuplas a la complejidad? \\ 
¿Cuál es la complejidad teórica en el peor caso, y en el promedio?  \\

\maketitle
\section{algoritmos de resolución de Knapsack} 

A continuación se presentaran las siguientes resoluciones del problema. Y posteriormente analizaremos su complejidad, 

\subsubsection{Algoritmo de fuerza bruta}

Próximamente en latex sabita:












\end{document}
