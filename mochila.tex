% !TEX TS-program = pdflatex
% !TEX encoding = UTF-8 Unicode

% This is a simple template for a LaTeX document using the "article" class.
% See "book", "report", "letter" for other types of document.

% align equations to the left
% use larger type; default would be 10pt
\documentclass[fleqn, 11pt]{article}

\usepackage[utf8]{inputenc} % set input encoding (not needed with XeLaTeX)

%%% Examples of Article customizations
% These packages are optional, depending whether you want the features they
% provide.
% See the LaTeX Companion or other references for full information.

%%% PAGE DIMENSIONS
\usepackage{geometry} % to change the page dimensions
\geometry{a4paper} % or letterpaper (US) or a5paper or....
% for example, change the margins to 2 inches all round
% \geometry{margin=2in}
% set up the page for landscape
% \geometry{landscape}
%   read geometry.pdf for detailed page layout information

% support the \includegraphics command and options
\usepackage{graphicx}

% Activate to begin paragraphs with an empty line rather than an indent
\usepackage[parfill]{parskip}

%%% PACKAGES
% for much better looking tables
\usepackage{booktabs}
% for better arrays (eg matrices) in maths
\usepackage{array}
% very flexible & customisable lists (eg. enumerate/itemize, etc.)
\usepackage{paralist}
% adds environment for commenting out blocks of text & for better verbatim
\usepackage{verbatim}
% make it possible to include more than one captioned figure/table in a single
% float
\usepackage{subfig}
% These packages are all incorporated in the memoir class to one degree or
% another...

%%% HEADERS & FOOTERS
\usepackage{fancyhdr} % This should be set AFTER setting up the page geometry
\pagestyle{fancy} % options: empty , plain , fancy
\renewcommand{\headrulewidth}{0pt} % customise the layout...
\lhead{}\chead{}\rhead{}
\lfoot{}\cfoot{\thepage}\rfoot{}

%%% SECTION TITLE APPEARANCE
\usepackage{sectsty}
% (See the fntguide.pdf for font help)
\allsectionsfont{\sffamily\mdseries\upshape}
% (This matches ConTeXt defaults)

%%% ToC (table of contents) APPEARANCE
% Put the bibliography in the ToC
\usepackage[nottoc, notlof, notlot]{tocbibind}
% Alter the style of the Table of Contents
\usepackage[titles, subfigure]{tocloft}
\renewcommand{\cftsecfont}{\rmfamily\mdseries\upshape}
\renewcommand{\cftsecpagefont}{\rmfamily\mdseries\upshape} % No bold!

\usepackage{calc}
\usepackage{lmodern}
\usepackage{amssymb}
\usepackage{amsmath}
\usepackage{mathdots}
\usepackage{enumerate}

\overfullrule=2em

%%% END Article customizations

%%% The "real" document content comes below...

\title{Algoritmos y Estructuras de Datos 3: Gloppi Ya}
\author{Ignacio E. Losiggio, Federico A. Sabatini}
% Activate to display a given date or no date (if empty), otherwise the current
% date is printed 
%\date{}

\begin{document}

\maketitle
\section{Presentación de Knapsack Problem} 

\subsubsection{
\textit{"El problema de la mochila (KP), consiste en asignar un conjunto de items con un peso y valor asociado a una mochila limitada por el peso que puede tolerar, buscando maximizar el valor de la misma."} }


\textbf{Formulacion del problema:}

Dado un espacio de tamaño W (la mochila), y un conjunto de tuplas $(x_i, y_i)$ con pesos $(x_1, x_2 \dots x_n)$ \\
y precios $(y_1, y_2 \dots y_n)$ ambos enteros no negativos, hallar un subconjunto de indices que \\
maximize el valor $z_{out} \in\mathbb{E} \geq 0$, de la siguiente forma:

\begin{itemize}
	\item $z_{out} $ = max $\sum \{ y_i : i \in indices \} $ \\ (subconjunto cuya sumatoria maximiza el precio)
	\item $\sum \{ x_i : i \in indices \} \le W $ \\ (subconjunto cuyos pesos no exceda el tamaño de W)
\end{itemize}

\subsubsection{Se plantea el siguiente ejemplo del problema:}

\begin{enumerate}[A.]

\item
\begin{tabular}{l l}
	peso (size) & precio \\ \toprule
    20 & 50 \\
    10 & 5  \\
    30 & 35  \\
\end{tabular}
con W = 30. \\

Res: $z_{out}$ es 55 y el subconjunto que maximiza el valor son las tuplas [(20,50) y (10,5)] \\

Se destaca en este caso que solo existen cuatro formas posibles de armar la mochila,
las otras forma serían usar tuplas singulares, (20,50), (10,5), (30,35), nada inhibe armar una mochila que no utilize todo su peso. Notar que el valor que maximiza puede encontrarse en cualquier subconjunto del conjunto de tuplas. 



\item
\begin{tabular}{l l}
	peso (size) & precio \\ \toprule
    45 & 150 \\
    20 & 50 \\
    20  & 40 \\
    20 & 60
    

\end{tabular}
con W = 60. \\

Res: $z_{out}$ es 150 y los subconjunto que maximizan el valor son la tupla [(45,150)] y el subconjunto
[(20, 50)] y [(20, 40), (20, 60)] \\

Se observa, que puede suceder que más de un solo subconjunto sea máximo.

\end{enumerate}

\subsubsection{Pautas de diseño utilizadas en este exámen:}

La idea en este estudio, será demostrar diferentes algoritmos que cumplen con las condiciones propuestas por KP, así como comparar la eficiencia de los mismos, Para esto se definiran diferentes entornos de prueba, en los cuales se someterá a los algoritmos a superar experimentos generados, \\

Para cada prueba de un experimento se forzara al algoritmo a resolverlo por lo menos 50 veces, y se determinará su velocidad para resolver el problema, con el promedio de las ejecuciones, de esa manera nos aseguraremos que si el algoritmo tiene un buen desempeño en resolver dicha prueba, sea altamente probable que se asemeje a la velocidad real.  \\

Para cada algoritmo planteado entonces se buscará resolver las siguientes preguntas:\\

¿Cómo es la resolución del algoritmo?  \\
¿Cómo afectan la variable W y la cantidad de N-tuplas a la complejidad? \\ 
¿Existen cotas de complejidad, o podas realizables?  \\
¿Cuál es la complejidad teórica en el peor caso, y en el promedio?  \\
¿Cuál es su impacto en la memoria?  \\

\maketitle
\section{algoritmos de resolución de Knapsack} 

A continuación se presentaran las siguientes resoluciones del problema. Y posteriormente analizaremos su complejidad, 

\subsubsection{Algoritmo de fuerza bruta}

Próximamente en latex sabita:












\end{document}
